\documentclass[a4paper,3.3mm]{article}
% to do:
% pass measurements to parts of metric system
% https://tex.stackexchange.com/questions/8260/what-are-the-various-units-ex-em-in-pt-bp-dd-pc-expressed-in-mm
% all corrections
% sort template
% MAKE template from this and the notes

% Encoding
\usepackage[english]{babel}
\usepackage[utf8]{inputenc}
\usepackage[T1]{fontenc}
\usepackage{lmodern}

% Paper geometry
\usepackage{geometry}
\usepackage{fancyhdr}
\usepackage{anysize}

% Miscellaneous
\usepackage[shortlabels]{enumitem}
\usepackage[hidelinks]{hyperref}
\usepackage{xifthen}
\usepackage{xparse}

% Mathematics & symbols
\usepackage{amsthm, amsmath, amsfonts, amssymb}
\usepackage{physics}
% \usepackage{esdiff}
\usepackage{mathtools}
\usepackage{float}
\usepackage{braket}
\usepackage{bm}
\usepackage{eurosym, gensymb}

% Line formatting ->? miscellaneous
\usepackage{textcomp}
\usepackage{txfonts}
\usepackage{lineno}

% Figures
\usepackage{hhline}
\usepackage{graphicx}
\usepackage{color}
\usepackage{subfigure}
\usepackage{multicol, multirow}
\usepackage{booktabs, cellspace, adjustbox}
\usepackage{caption}


% Custom commands

% Shorter \mathrm
\let\nm\mathrm
\let\mathrm\relax

% Shorter \textit
\renewcommand{\it}[1]{\textit{#1}}

% Shorter \emph\textbf
\newcommand{\embf}[1]{\emph{\textbf{#1}}}

% \ref with custom text, using package hyperref
\newcommand{\cref}[2][\textbf{[?!]}]{\hyperref[#2]{#1}}

% Real number set
\newcommand{\R}[1]{\varmathbb{R}^{#1}}

% Fractions of pi -> introduce optional argument for numerator
\newcommand{\fpi}[1]{\frac{\pi}{#1}}


% Differential letter
\newcommand*\diff{\mathop{}\!\mathrm{d}}
% if using esdiff, rename this previous command to \renewcommand*
\newcommand*\Diff[1]{\mathop{}\!\mathrm{d^{#1}}}

% Partial differential letter
\newcommand*\pdiff{\mathop{}\!\partial}
\newcommand*\pDiff[1]{\mathop{}\!\partial^{#1}}

% Inexact differential letter
\newcommand*\idiff{\mathop{}\!\text{\dj}}
\newcommand*\iDiff[1]{\mathop{}\!\text{\dj}^{#1}}
\let\dbar\dj

% Integral
\let\integral\int

% Class k (function, curve, etc.) in differentiation context
\newcommand{\dclass}[1]{\mathcal{C}^{#1}}

% \left( \right) container
\newcommand{\parn}[1]{\left(#1\right)}

% \left[ \right] container
\newcommand{\sbra}[1]{\left[#1\right]}

% \left{ \right} container
\newcommand{\cbra}[1]{\left\{#1\right\}}



% Page geometry (margins, etc.)

\headheight = 0mm
\marginsize{14mm}{14mm}{21mm}{21mm} % left right top bottom
\parindent = 0mm
\parskip = 7mm
\marginparwidth = 21mm
\marginparsep = 7mm

\newcommand{\pindent}{\parindent=5.5mm\indent\parindent=0mm\relax}

% Document information

\title{Análisis de consistencia y estabilidad del esquema Dufort-Frankel para la ecuación de difusión}
\author{Mario García Cajade}
\date{}

\pagestyle{fancy}
\lhead{}
\chead{}
\rhead{Mario García Cajade}
\renewcommand{\headrulewidth}{0mm}
\lfoot{}
\cfoot{}
\rfoot{}
\renewcommand{\footrulewidth}{0mm}


% Document begin

\begin{document}

{
\centering
\vspace*{-14mm}
\fontsize{8mm}{12mm}\selectfont%
Análisis de consistencia y estabilidad del esquema Dufort-Frankel para la ecuación de difusión\\[-4mm]
\[ \frac{\partial T}{\partial t} = \alpha\,\frac{\partial^{2} T}{\partial x^{2}} \]
}

\raggedright
\fontsize{5.25mm}{7mm}\selectfont
\section*{\fontsize{7mm}{10.5mm}\selectfont Esquema}

La ecuación del esquema Dufort-Frankel es la siguiente:

\begin{equation}
    \frac{T_{i}^{n+1} - T_{i}^{n-1}}{2 \Delta t} = \alpha\,\frac{T_{i-1}^{n} - \parn{ T_{i}^{n+1} + T_{i}^{n-1}} + T_{i+1}^{n}}{\Delta x^{2} }
    \tag{DF}
\end{equation}

\section*{\fontsize{7mm}{10.5mm}\selectfont Consistencia}

Utilizando expansión de Taylor se obtiene fácilmente

\[
    \parn{ \frac{\partial T}{\partial t} }_{j}^{n} - \alpha\,\parn{ \frac{\partial^{2} T}{\partial x^{2}} }_{j}^{n} + \alpha\,\frac{\Delta t^{2}}{\Delta x^{2}}\,\parn{ \frac{\partial T}{\partial t} }_{j}^{n} + O\,\parn{ \Delta t^{2},\, \Delta x^{2} }
\]

Para que exista consistencia debe cumplirse \( \frac{\Delta t}{\Delta x} \rightarrow 0 \)
también a medida que se mejora la cuadrícula del problema (el número de particiones aumenta y consecuentemente el valor de 
\(\Delta x\) disminuye), por lo que el esquema será consistente para \( \Delta t << \Delta x \).

\newpage
\section*{\fontsize{7mm}{10.5mm}\selectfont Estabilidad}

Utilizando el método de Von Neumann se buscan soluciones de la forma

\[
    T_{j}^{n} = \kappa\,T_{0}\,e^{i\,j\,k\,\Delta x}
\]

siendo \( k\,\Delta x \) la longitud de onda (\( k \geq 1 \)), \( T_{0} \) el valor inicial de la amplitud y \( \kappa \)
el factor de amplificación entre intervalos temporales.

En general \( \kappa = \kappa \parn{ k\,\Delta x } \) y tenemos que calcularlo sabiendo que

\[
    T_{j}^{n+1} = \kappa\,T_{j}^{n}; \quad T_{j}^{n-1} = \frac{1}{\kappa}\,T_{j}^{n}; \quad T_{j \pm 1}^{n} = T_{j}^{n}\,e^{\pm i\,k\,\Delta x}.
\]

La solución propia del esquema se convierte en

\[
    \parn{ \kappa - \frac{1}{\kappa} } = 2\,\zeta\,\parn{ 2 \cos(\lambda) - \kappa - \frac{1}{\kappa} }
\]

con \( \lambda = k\,\Delta x \) y \( \zeta = \alpha\,\frac{\Delta t}{\Delta x^{2}} \). Resultan entonces dos raíces

\[
    \kappa^{\pm} = \frac{ \zeta\,\cos(\lambda) \pm \sqrt{ 1 - \zeta^{2}\,\sin^{2}(\lambda) } }{ 1 + \zeta }
\]

cumpliéndose \( \abs{\kappa^{\pm}} \leq 1 \) para todo \(\lambda\) y para todo \(\zeta\), por lo que el esquema es
incondicionalmente estable.

\end{document}